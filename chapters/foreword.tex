\addchap{Foreword}

This wonderful book presents the best of modern linguistics: a very attractive combination of rigour, information, insight, and fun. As chair of the UK Linguistics Olympiad, I commend it strongly to anyone with an interest in linguistics olympiads. I also congratulate Vlad Neacșu on producing such a useful guide and thank him for making its translation freely available to anyone who can read English.

I listed four attractive qualities of modern linguistics: rigour, information, insight, and fun. All four make linguistics an outstanding candidate for inclusion in the school curriculum, so I should like to take this opportunity to justify my claims.

\textit{Rigour}\ is obvious. Our champions tend strongly to be mathematicians with a gift for spotting abstract patterns and building a logical chain of reasoning. This book takes the reader by the hand through some of the most daunting problems available, showing how each one allows just one correct solution, and sometimes just one possible route to that solution. Constructing one of these problems is an intellectual triumph in itself, so they testify to three sources of rigour: in the problem-creator, in the problem-solver, and in the problem-explainer. The explanations are brilliantly clear and untechnical.
 
The trouble with rigour is that for some people it doesn't come naturally. It is all too easy to find olympiad problems that look, at first sight, as though an analysis is impossible. Where to start? How to get a foothold on the data? This is where the book is particularly strong, because, after introducing a simple classification of problems, it provides tactics for each type of problem: counting, drawing graphs, constructing tables and so on. Just what a beginner needs, and a wonderful source of confidence. So instead of relying solely on native ability, we approach problems with a toolbag full of helpful ideas and skills.
 
Rigour is an important attribute of linguistic analysis because this is what puts linguistics on the same level as mathematics. Our schools in the UK give high status to mathematics and science, but low status to language; one of the arguments for mathematics is its rigour and its effect on thinking skills, so we now have a similar argument for language – but only if we stop thinking about language simply as a useful skill, and start viewing it as a worthy object of rigorous study.
 
\textit{Information}\ is available in spadefuls. The book is an elementary introduction to the variety of human language, with plenty of opportunities for exploring this variety in datasets from an amazing range of languages. All the problems were built by professional linguisticians from across the world, and, since every language is unique, they all take us to the frontiers of research. The book takes us through the many variables that are familiar to linguistic typologists, but without straying into high theory – everything is tied firmly to the evidence in the data.

Every problem in this book introduces a language system which, in some respect, is different from English. The differences cover all the levels of language, from writing systems and phonetics to semantics, and in every problem we discover the regularities that lie behind the apparently chaotic data. The variety that emerges is truly stunning and must be part of the reason why children find linguistics olympiads so gripping.

These problems manifest in very concrete ways the claim that learning a foreign language takes us into the different mental world of its speakers. But unlike a foreign-language lesson, we reach that world and explore it in just a few minutes and without any drills or memorisation. Lip service is often paid in educational circles to the goal of Language Awareness, but this goal conflicts too often with that of teaching only for communication. Language teaching would be much more successful if our children spent as much time on linguistics olympiads as they do in communicative classrooms where they fail to learn even a single language.
 
\textit{Insight}\ takes us into the workings of language. We all have expert knowledge of at least one language, but without linguistics, we don't know how it works. Struggling through one of these problems gives us insight into one small part of the great machine of language; for example, any script problem forces us to confront issues such as syllable boundaries, classification of sounds, morphological and semantic analysis, and all the possible relations between characters and linguistic units. It's all too easy for a child to believe that a word's pronunciation and spelling are the same thing, and to be surprised that \cmubdata{though} only has two sounds. Any activity, such as an olympiad problem, that problematises this simple view is to be welcomed.
 
The main insight promoted here is what we call the architecture of language – how the total structure is divided into the traditional levels of phonetics, phonology, morphology, syntax, and semantics. This rather sophisticated view is the more or less uncontroversial basis for linguistics, and should be part of the worldview of any educated citizen; but while we are waiting for our schools to catch up, the olympiad is the main, or even only, tool for teaching it.
 
And last, but by no means least, we have \textit{fun}.  The popularity of the linguistics olympiad shows that school children enjoy grappling with linguistic analysis. Our teachers repeatedly report that children are excited, engaged, and enthusiastic; and many thousands of them come back for more, year after year (in 2023 the UK Olympiad attracted over 5,000 competitors, all willing volunteers). Our problems offer the abstract intellectual challenge of sudoku combined with the complex strategic thinking of chess and the excitement of exploring a foreign country.
 
This enthusiasm for linguistics olympiads is all the more remarkable given not only the total absence of linguistics from the school curriculum but also the deep unpopularity of foreign languages among our school children, who tend to find them both difficult and boring. Maybe it's time for language teachers to pay more attention to the system of the target language as something that the learners might enjoy exploring.

Which brings us back to this book. Children enjoy cracking codes and exploring the intellectual territory behind the codes, but their enjoyment depends on making progress. Nobody is inspired by the soul-destroying experience of staring at a page of data for half an hour without making any sense of it at all; so children need intellectual tools to help them on their way. This book is just what they need.

\begin{flushright}--- Richard Hudson \\ 
\textit{Chair of the UK Linguistics Olympiad \\
Emeritus Professor at University College London} \end{flushright}
