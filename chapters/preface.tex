\addchap{\lsPrefaceTitle}

This book is a training guide (handbook) for the linguistics olympiads. It is the first such endeavour both nationally and internationally, since, at the moment, the only published materials contain collections of problems, rather than material to guide students and teachers, by presenting the theoretical framework and linguistic phenomena needed when solving linguistics problems, as well as different methods which can be applied when solving these problems. 
 
All chapters follow the same structure: a short theoretical introduction followed by a detailed explanation of the less-known linguistic phenomena and further depicting these phenomena through one or more linguistics problems, solved step by step. At the end of each chapter there is a list of practice problems from national and international linguistics olympiads, accompanied by the answers, solutions, and, in some cases, additional explanations or discussions. 

It is worth mentioning that this guide is not aimed at a certain level, but rather attempts to cover all relevant linguistic information from the very beginners all the way to those students who are training for the International Linguistics Olympiad (henceforth, IOL). The chapters and subchapters follow a logical order from a linguistic perspective, which does not necessarily correspond to the level of difficulty. This was done in order to attempt, as much as possible, a gradual display of terminology to avoid having to use certain linguistic terms before they are introduced and explained in detail.  

When it comes to actively employing this book, it is best that students (as well as anyone else who wishes to start considering this type of problem) first attempt to solve the problems without checking the suggested solution and then read it, step by step, and check whether they reached the same results, the same rules, the same translations. Moreover, the way problems are approached in this guide is not necessarily “absolute” and many other approaches may exist which yield the same result. Therefore, I attempted as much as possible to approach each problem in a different manner, showcasing at the same time certain “tricks” or methods that might prove useful when attempting new problems. The problems in the “Practice problems” sections are mostly placed in order of difficulty, starting from problems suitable for beginners and ending with problems comparable to those from international olympiads. As such, it is perfectly normal that some problems may seem extremely hard to solve and thus you might need to take a glance at the solution in order to get some hints. Once you have solved the problem, the most important thing is to attempt to fully understand the problem and the solution, both in terms of the linguistic phenomena that are showcased, as well as the way that you could approach similar problems in the future. This is a complex guide that presents the most common phenomena that have already been featured in linguistics problems, but it is far from being a complete guide, since new languages, features, phenomena, and even solving approaches are always discovered (and even emerging).

One of the main attributes of linguistics problems is that they can be solved without any additional information (solely based on the given data). Working through this training guide needs to be continuously complemented by solving problems in order to be able to perpetually practice the things you have learnt and to keep up with this ever-evolving competition. 


Experienced linguists might notice that some aspects of the material presented herein do not always follow standard academic practices. This is because this book is not meant to be an academic work, but rather aimed at young people discovering things about language and how to approach linguistics problems. For example, readers may notice that figures are only labelled as such when they form part of the discursive text and not when they are part of the problem text or its solution. Similarly, certain linguistic concepts have been simplified in order to maximise the accessibility of the text. 


\textit{Please read this book in the spirit in which it was written!}

\begin{flushright} 
--- Vlad A. Neacșu
\end{flushright}

\bigskip

\begin{tblsfilled}{}
For any suggestions, observations, or other inquiries, the author can be contacted directly at \href{mailto:vlad.neacsu2009@gmail.com}{vlad.neacsu2009@gmail.com}.
\end{tblsfilled}
