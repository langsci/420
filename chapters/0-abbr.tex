\addchap{Abbreviations and notes}
\addsec{Abbreviations used to indicate the source of the problems}

\begin{tabular}{@{}ll@{}}
    \APLOAbbr  &  \APLOName\\
    \ElementyAbbr  &  \ElementyName \\ 
    \HKLOAbbr  & \HKLOName\\ 
    \IOLAbbr  &  \IOLName \\ 
    \TurLomAbbr  &  \TurLomName\\ 
    \NACLOAbbr  &  \NACLOName\\ 
    \CLOAbbr  &  \CLOName\\ 
    \JOLAbbr  &  \JOLName\\ 
    \LLOAbbr  &  \LLOName\\ 
    \RoLOAbbr  &  \RoLOName\\ 
    \MSKAbbr  &  \MSKName\\ 
    \UkrLOAbbr  &  \UkrLOName\\ 
    \PLOAbbr  &  \PLOName\\ 
    \PrincetonAbbr  &  \PrincetonName\\ 
    \UKLOAbbr  &  \UKLOName\\ 
\end{tabular}\bigskip\\
\noindent The year mentioned in the beginning of each problem refers to the year in which the problem was used, synced with the corresponding \IOLAbbr\ edition, e.g., if the problem was used in the \UKLOAbbr\ in the academic year 2002--2003, the problem will be marked as \LOYear{\UKLOAbbr}{2023}.

\addsec{Note about problems}

Problems from APLO, HKLO, IOL, NACLO, PLO, Princeton, and UKLO were used in the original English version. All the other problems were translated by the author.

Most of the problems in the present volume were previously used in national and international competitions and were based on grammar books, grammar sketches, or dictionaries of that language, sometimes complemented by information found on the internet, official websites and so on. For example, Problem 4.6 was based on \citet{Li1985}, while Problem 6.13 was based on \citet{Refsing1986} and \citet{Shibatani1990}.

It is important to mention that these problems were created for the purpose of linguistics contests and to test the students' logical reasoning abilities. While the data are definitely not fictitious, some small simplifications such as changes in orthography may have been made in order to make the problem accessible to the target group.
